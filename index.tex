% Options for packages loaded elsewhere
% Options for packages loaded elsewhere
\PassOptionsToPackage{unicode}{hyperref}
\PassOptionsToPackage{hyphens}{url}
\PassOptionsToPackage{dvipsnames,svgnames,x11names}{xcolor}
%
\documentclass[
  letterpaper,
  DIV=11,
  numbers=noendperiod]{scrartcl}
\usepackage{xcolor}
\usepackage{amsmath,amssymb}
\setcounter{secnumdepth}{-\maxdimen} % remove section numbering
\usepackage{iftex}
\ifPDFTeX
  \usepackage[T1]{fontenc}
  \usepackage[utf8]{inputenc}
  \usepackage{textcomp} % provide euro and other symbols
\else % if luatex or xetex
  \usepackage{unicode-math} % this also loads fontspec
  \defaultfontfeatures{Scale=MatchLowercase}
  \defaultfontfeatures[\rmfamily]{Ligatures=TeX,Scale=1}
\fi
\usepackage{lmodern}
\ifPDFTeX\else
  % xetex/luatex font selection
\fi
% Use upquote if available, for straight quotes in verbatim environments
\IfFileExists{upquote.sty}{\usepackage{upquote}}{}
\IfFileExists{microtype.sty}{% use microtype if available
  \usepackage[]{microtype}
  \UseMicrotypeSet[protrusion]{basicmath} % disable protrusion for tt fonts
}{}
\makeatletter
\@ifundefined{KOMAClassName}{% if non-KOMA class
  \IfFileExists{parskip.sty}{%
    \usepackage{parskip}
  }{% else
    \setlength{\parindent}{0pt}
    \setlength{\parskip}{6pt plus 2pt minus 1pt}}
}{% if KOMA class
  \KOMAoptions{parskip=half}}
\makeatother
% Make \paragraph and \subparagraph free-standing
\makeatletter
\ifx\paragraph\undefined\else
  \let\oldparagraph\paragraph
  \renewcommand{\paragraph}{
    \@ifstar
      \xxxParagraphStar
      \xxxParagraphNoStar
  }
  \newcommand{\xxxParagraphStar}[1]{\oldparagraph*{#1}\mbox{}}
  \newcommand{\xxxParagraphNoStar}[1]{\oldparagraph{#1}\mbox{}}
\fi
\ifx\subparagraph\undefined\else
  \let\oldsubparagraph\subparagraph
  \renewcommand{\subparagraph}{
    \@ifstar
      \xxxSubParagraphStar
      \xxxSubParagraphNoStar
  }
  \newcommand{\xxxSubParagraphStar}[1]{\oldsubparagraph*{#1}\mbox{}}
  \newcommand{\xxxSubParagraphNoStar}[1]{\oldsubparagraph{#1}\mbox{}}
\fi
\makeatother

\usepackage{color}
\usepackage{fancyvrb}
\newcommand{\VerbBar}{|}
\newcommand{\VERB}{\Verb[commandchars=\\\{\}]}
\DefineVerbatimEnvironment{Highlighting}{Verbatim}{commandchars=\\\{\}}
% Add ',fontsize=\small' for more characters per line
\usepackage{framed}
\definecolor{shadecolor}{RGB}{241,243,245}
\newenvironment{Shaded}{\begin{snugshade}}{\end{snugshade}}
\newcommand{\AlertTok}[1]{\textcolor[rgb]{0.68,0.00,0.00}{#1}}
\newcommand{\AnnotationTok}[1]{\textcolor[rgb]{0.37,0.37,0.37}{#1}}
\newcommand{\AttributeTok}[1]{\textcolor[rgb]{0.40,0.45,0.13}{#1}}
\newcommand{\BaseNTok}[1]{\textcolor[rgb]{0.68,0.00,0.00}{#1}}
\newcommand{\BuiltInTok}[1]{\textcolor[rgb]{0.00,0.23,0.31}{#1}}
\newcommand{\CharTok}[1]{\textcolor[rgb]{0.13,0.47,0.30}{#1}}
\newcommand{\CommentTok}[1]{\textcolor[rgb]{0.37,0.37,0.37}{#1}}
\newcommand{\CommentVarTok}[1]{\textcolor[rgb]{0.37,0.37,0.37}{\textit{#1}}}
\newcommand{\ConstantTok}[1]{\textcolor[rgb]{0.56,0.35,0.01}{#1}}
\newcommand{\ControlFlowTok}[1]{\textcolor[rgb]{0.00,0.23,0.31}{\textbf{#1}}}
\newcommand{\DataTypeTok}[1]{\textcolor[rgb]{0.68,0.00,0.00}{#1}}
\newcommand{\DecValTok}[1]{\textcolor[rgb]{0.68,0.00,0.00}{#1}}
\newcommand{\DocumentationTok}[1]{\textcolor[rgb]{0.37,0.37,0.37}{\textit{#1}}}
\newcommand{\ErrorTok}[1]{\textcolor[rgb]{0.68,0.00,0.00}{#1}}
\newcommand{\ExtensionTok}[1]{\textcolor[rgb]{0.00,0.23,0.31}{#1}}
\newcommand{\FloatTok}[1]{\textcolor[rgb]{0.68,0.00,0.00}{#1}}
\newcommand{\FunctionTok}[1]{\textcolor[rgb]{0.28,0.35,0.67}{#1}}
\newcommand{\ImportTok}[1]{\textcolor[rgb]{0.00,0.46,0.62}{#1}}
\newcommand{\InformationTok}[1]{\textcolor[rgb]{0.37,0.37,0.37}{#1}}
\newcommand{\KeywordTok}[1]{\textcolor[rgb]{0.00,0.23,0.31}{\textbf{#1}}}
\newcommand{\NormalTok}[1]{\textcolor[rgb]{0.00,0.23,0.31}{#1}}
\newcommand{\OperatorTok}[1]{\textcolor[rgb]{0.37,0.37,0.37}{#1}}
\newcommand{\OtherTok}[1]{\textcolor[rgb]{0.00,0.23,0.31}{#1}}
\newcommand{\PreprocessorTok}[1]{\textcolor[rgb]{0.68,0.00,0.00}{#1}}
\newcommand{\RegionMarkerTok}[1]{\textcolor[rgb]{0.00,0.23,0.31}{#1}}
\newcommand{\SpecialCharTok}[1]{\textcolor[rgb]{0.37,0.37,0.37}{#1}}
\newcommand{\SpecialStringTok}[1]{\textcolor[rgb]{0.13,0.47,0.30}{#1}}
\newcommand{\StringTok}[1]{\textcolor[rgb]{0.13,0.47,0.30}{#1}}
\newcommand{\VariableTok}[1]{\textcolor[rgb]{0.07,0.07,0.07}{#1}}
\newcommand{\VerbatimStringTok}[1]{\textcolor[rgb]{0.13,0.47,0.30}{#1}}
\newcommand{\WarningTok}[1]{\textcolor[rgb]{0.37,0.37,0.37}{\textit{#1}}}

\usepackage{longtable,booktabs,array}
\usepackage{calc} % for calculating minipage widths
% Correct order of tables after \paragraph or \subparagraph
\usepackage{etoolbox}
\makeatletter
\patchcmd\longtable{\par}{\if@noskipsec\mbox{}\fi\par}{}{}
\makeatother
% Allow footnotes in longtable head/foot
\IfFileExists{footnotehyper.sty}{\usepackage{footnotehyper}}{\usepackage{footnote}}
\makesavenoteenv{longtable}
\usepackage{graphicx}
\makeatletter
\newsavebox\pandoc@box
\newcommand*\pandocbounded[1]{% scales image to fit in text height/width
  \sbox\pandoc@box{#1}%
  \Gscale@div\@tempa{\textheight}{\dimexpr\ht\pandoc@box+\dp\pandoc@box\relax}%
  \Gscale@div\@tempb{\linewidth}{\wd\pandoc@box}%
  \ifdim\@tempb\p@<\@tempa\p@\let\@tempa\@tempb\fi% select the smaller of both
  \ifdim\@tempa\p@<\p@\scalebox{\@tempa}{\usebox\pandoc@box}%
  \else\usebox{\pandoc@box}%
  \fi%
}
% Set default figure placement to htbp
\def\fps@figure{htbp}
\makeatother





\setlength{\emergencystretch}{3em} % prevent overfull lines

\providecommand{\tightlist}{%
  \setlength{\itemsep}{0pt}\setlength{\parskip}{0pt}}



 


\KOMAoption{captions}{tableheading}
\makeatletter
\@ifpackageloaded{caption}{}{\usepackage{caption}}
\AtBeginDocument{%
\ifdefined\contentsname
  \renewcommand*\contentsname{Table of contents}
\else
  \newcommand\contentsname{Table of contents}
\fi
\ifdefined\listfigurename
  \renewcommand*\listfigurename{List of Figures}
\else
  \newcommand\listfigurename{List of Figures}
\fi
\ifdefined\listtablename
  \renewcommand*\listtablename{List of Tables}
\else
  \newcommand\listtablename{List of Tables}
\fi
\ifdefined\figurename
  \renewcommand*\figurename{Figure}
\else
  \newcommand\figurename{Figure}
\fi
\ifdefined\tablename
  \renewcommand*\tablename{Table}
\else
  \newcommand\tablename{Table}
\fi
}
\@ifpackageloaded{float}{}{\usepackage{float}}
\floatstyle{ruled}
\@ifundefined{c@chapter}{\newfloat{codelisting}{h}{lop}}{\newfloat{codelisting}{h}{lop}[chapter]}
\floatname{codelisting}{Listing}
\newcommand*\listoflistings{\listof{codelisting}{List of Listings}}
\makeatother
\makeatletter
\makeatother
\makeatletter
\@ifpackageloaded{caption}{}{\usepackage{caption}}
\@ifpackageloaded{subcaption}{}{\usepackage{subcaption}}
\makeatother
\usepackage{bookmark}
\IfFileExists{xurl.sty}{\usepackage{xurl}}{} % add URL line breaks if available
\urlstyle{same}
\hypersetup{
  pdftitle={Simulation Challenge},
  colorlinks=true,
  linkcolor={blue},
  filecolor={Maroon},
  citecolor={Blue},
  urlcolor={Blue},
  pdfcreator={LaTeX via pandoc}}


\title{Simulation Challenge}
\usepackage{etoolbox}
\makeatletter
\providecommand{\subtitle}[1]{% add subtitle to \maketitle
  \apptocmd{\@title}{\par {\large #1 \par}}{}{}
}
\makeatother
\subtitle{Generative Models and Monte Carlo Simulation}
\author{}
\date{}
\begin{document}
\maketitle


\section{🎲 Simulation Challenge - Monte Carlo
Analysis}\label{simulation-challenge---monte-carlo-analysis}

\subsection{Challenge Overview}\label{challenge-overview}

\textbf{Your Mission:} Create a comprehensive Quarto document that
simulates one or two investment strategies, analyzes the results, and
demonstrates your ability to present counter-intuitive findings
compellingly. Then render the document to HTML and deploy it via GitHub
Pages from a new repository called ``simulationChallenge.''

\subsection{The Investment Game 🎯}\label{the-investment-game}

\subsubsection{Original Game Strategy}\label{original-game-strategy}

Imagine you are offered the following game and given a \$1,000 budget in
a special account to play the game: I will flip a coin, and if it comes
up heads, we increase your account's balance by 50\%; if it comes up
tails, we reduce your account's balance by 40\%. We are not only doing
this once, but we will do it once per year until you turn 55. When you
turn 55, you will receive the balance in your account.

\subsection{Challenge Requirements 📋}\label{challenge-requirements}

\subsubsection{Questions to Answer for 75\% Grade on
Challenge}\label{questions-to-answer-for-75-grade-on-challenge}

\begin{enumerate}
\def\labelenumi{\arabic{enumi}.}
\item
  \textbf{Expected Value Analysis:} What is the ``expected value'' of
  your account balance after 1 coin flip for the original game?
\item
  \textbf{Expectation vs.~Reality:} Is the expected value positive or
  negative? Do you expect your account to be worth more or less than
  \$1,000 based on this result?
\item
  \textbf{Single Simulation:} Run one simulation showing the dynamics of
  your account balance over time. Make an object-oriented matplotlib OR
  ggplot2 plot showing your simulated account balance over time (i.e.~as
  you age). Comment on the results, are you happy?
\end{enumerate}

\subsubsection{Questions to Answer for 85\% Grade on
Challenge}\label{questions-to-answer-for-85-grade-on-challenge}

\begin{enumerate}
\def\labelenumi{\arabic{enumi}.}
\setcounter{enumi}{3}
\tightlist
\item
  \textbf{Multiple Simulations:} Run 100 simulations modelling the
  dynamics of your account balance over time. Make an object-oriented
  matplotlib OR ggplot2 plot showing a probability distribution of the
  100 simulated account balance at age 55. Comment on the results, are
  you happy? Why or why not?
\end{enumerate}

\subsubsection{Questions to Answer for 95\% Grade on
Challenge}\label{questions-to-answer-for-95-grade-on-challenge}

\begin{enumerate}
\def\labelenumi{\arabic{enumi}.}
\setcounter{enumi}{4}
\tightlist
\item
  \textbf{Probability Analysis:} Based on the 100 simulations above,
  what is the probability that your account balance will be greater than
  \$1,000 at age 55?
\end{enumerate}

\subsubsection{Questions to Answer for 100\% Grade on
Challenge}\label{questions-to-answer-for-100-grade-on-challenge}

\begin{enumerate}
\def\labelenumi{\arabic{enumi}.}
\setcounter{enumi}{5}
\tightlist
\item
  \textbf{Strategy Comparison:} Run 100 simulations for the modified
  game strategy shown below. What is the probability that your account
  balance will be greater than \$10,000 at age 55? Is this probability
  higher or lower than the probability in the original game?
\end{enumerate}

\subsubsection{Modified Game Strategy}\label{modified-game-strategy}

Imagine you are offered the following game and given a \$1,000 budget in
a special account to play the game: I will flip a coin, and if it comes
up heads, we increase your bet by 50\%; if it comes up tails, we reduce
your bet by 40\%. You must bet exactly 50\% of your current account
balance on each flip, and this 50\% is locked in for each round. We are
not only doing this once, but we will do it once per year until you turn
55. When you turn 55, you will receive the balance in your account.

\subsection{Getting Started}\label{getting-started}

This document is now ready for you to add your simulation code and
analysis. You can start by adding R or Python code blocks to work
through the challenge questions.

\subsubsection{Example R Code Block}\label{example-r-code-block}

\begin{Shaded}
\begin{Highlighting}[]
\CommentTok{\# Set seed for reproducibility}
\FunctionTok{set.seed}\NormalTok{(}\DecValTok{123}\NormalTok{)}

\CommentTok{\# Simple example calculation}
\NormalTok{initial\_balance }\OtherTok{\textless{}{-}} \DecValTok{1000}
\NormalTok{gain\_rate }\OtherTok{\textless{}{-}} \FloatTok{0.5}
\NormalTok{loss\_rate }\OtherTok{\textless{}{-}} \FloatTok{0.4}

\CommentTok{\# Expected value calculation}
\NormalTok{expected\_value }\OtherTok{\textless{}{-}}\NormalTok{ initial\_balance }\SpecialCharTok{*}\NormalTok{ (}\FloatTok{0.5} \SpecialCharTok{*}\NormalTok{ (}\DecValTok{1} \SpecialCharTok{+}\NormalTok{ gain\_rate) }\SpecialCharTok{+} \FloatTok{0.5} \SpecialCharTok{*}\NormalTok{ (}\DecValTok{1} \SpecialCharTok{{-}}\NormalTok{ loss\_rate))}
\FunctionTok{print}\NormalTok{(}\FunctionTok{paste}\NormalTok{(}\StringTok{"Expected value after one flip:"}\NormalTok{, expected\_value))}
\end{Highlighting}
\end{Shaded}

\begin{verbatim}
[1] "Expected value after one flip: 1050"
\end{verbatim}

\subsubsection{Example Python Code
Block}\label{example-python-code-block}

\begin{Shaded}
\begin{Highlighting}[]
\CommentTok{\#| label: example{-}python}
\CommentTok{\#| echo: true}
\CommentTok{\#| eval: false}

\ImportTok{import}\NormalTok{ numpy }\ImportTok{as}\NormalTok{ np}

\CommentTok{\# Set seed for reproducibility}
\NormalTok{np.random.seed(}\DecValTok{123}\NormalTok{)}

\CommentTok{\# Simple example calculation}
\NormalTok{initial\_balance }\OperatorTok{=} \DecValTok{1000}
\NormalTok{gain\_rate }\OperatorTok{=} \FloatTok{0.5}
\NormalTok{loss\_rate }\OperatorTok{=} \FloatTok{0.4}

\CommentTok{\# Expected value calculation}
\NormalTok{expected\_value }\OperatorTok{=}\NormalTok{ initial\_balance }\OperatorTok{*}\NormalTok{ (}\FloatTok{0.5} \OperatorTok{*}\NormalTok{ (}\DecValTok{1} \OperatorTok{+}\NormalTok{ gain\_rate) }\OperatorTok{+} \FloatTok{0.5} \OperatorTok{*}\NormalTok{ (}\DecValTok{1} \OperatorTok{{-}}\NormalTok{ loss\_rate))}
\BuiltInTok{print}\NormalTok{(}\SpecialStringTok{f"Expected value after one flip: }\SpecialCharTok{\{}\NormalTok{expected\_value}\SpecialCharTok{\}}\SpecialStringTok{"}\NormalTok{)}
\end{Highlighting}
\end{Shaded}

\subsection{Next Steps}\label{next-steps}

\begin{enumerate}
\def\labelenumi{\arabic{enumi}.}
\tightlist
\item
  Add your simulation code for each question
\item
  Create visualizations showing your results
\item
  Provide clear interpretations of your findings
\item
  Deploy to GitHub Pages when complete
\end{enumerate}

Good luck with your simulation challenge! 🎲




\end{document}
